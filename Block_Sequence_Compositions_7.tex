\documentclass[preprint, 12pt]{elsarticle}
\usepackage{chemfig}
\usepackage{tikz}
\usepackage{graphicx}
\usepackage{amsmath, amssymb}
\setlength{\parindent}{0pt}
\usepackage{pgfplots}
\pgfplotsset{
	compat=1.3,
}
\pgfplotscreateplotcyclelist{line styles}{
	black,solid\\
	blue,dashed\\
	red,dotted\\
	orange,dashdotted\\
}

\newcommand*\GnuplotDefs{
	% set number of samples
	set samples 50;
	%
	%%% from <https://en.wikipedia.org/wiki/Normal_distribution>
	% cumulative distribution function (CDF) of normal distribution
	cdfn(x,mu,sd) = 0.5 * ( 1 + erf( (x-mu)/sd/sqrt(2)) );
	% probability density function (PDF) of normal distribution
	pdfn(x,mu,sd) = 1/(sd*sqrt(2*pi)) * exp( -(x-mu)^2 / (2*sd^2) );
	% PDF of a truncated normal distribution
	tpdfn(x,mu,sd,a,b) = pdfn(x,mu,sd) / ( cdfn(b,mu,sd) - cdfn(a,mu,sd) );
}
\usepackage{geometry}
\usepackage{mathtools}
\usepackage{tkz-berge}
\usetikzlibrary{automata}
\usetikzlibrary{arrows}
\usetikzlibrary{positioning,shapes,shadows,arrows}
\usetikzlibrary{shapes.geometric}
\usetikzlibrary{calendar,shadings}
\renewcommand*{\familydefault}{\sfdefault}
\colorlet{winter}{blue}
\colorlet{spring}{green!60!black}
\colorlet{summer}{orange}
\colorlet{fall}{red}
\newcount\mycount

\newcommand\shapeLarge{50mm}
\newcommand\shapeMedium{25mm}
\newcommand\shapeSmall{5mm}

\newcommand*{\xMin}{0}%
\newcommand*{\xMax}{6}%
\newcommand*{\yMin}{0}%
\newcommand*{\yMax}{6}%
\newcommand*{\zMax}{6}%
\newcommand*{\zMin}{0}%

\definecolor{colorwaveA}{RGB}{98,145,224}
\definecolor{colorwaveB}{RGB}{250,250,50}
\definecolor{colorwaveC}{RGB}{25,125,25}
\definecolor{colorwaveD}{RGB}{100,100,100}
\definecolor{colorwaveE}{RGB}{80,100,1}
\definecolor{colorwaveF}{RGB}{60,1,1}
\definecolor{colorwaveG}{RGB}{25,1,100}
\definecolor{colorwaveH}{RGB}{1,90,1}
\definecolor{colorwaveI}{RGB}{1,100,1}
\definecolor{colorwaveJ}{RGB}{1,1,1}


\tikzset{%
	shapeTriangle/.style={draw,shape=regular polygon,fill=colorwaveA,circular drop shadow,regular polygon sides=3,minimum size=\shapeSmall,inner sep=0pt,outer sep=0pt},
	shapeTriangle3/.style={shapeTriangle,fill=colorwaveD,circular drop shadow,shape border rotate=45},
	shapeTriangle4/.style={shapeTriangle,fill=colorwaveA,circular drop shadow,shape border rotate=90},
	shapeTriangle5/.style={shapeTriangle,fill=colorwaveB,shape border rotate=135},
	shapeTriangle6/.style={shapeTriangle,fill=colorwaveC,shape border rotate=180},
	shapeTriangle7/.style={shapeTriangle,fill=colorwaveE,shape border rotate=225},
	shapeTriangle8/.style={shapeTriangle,fill=colorwaveF,shape border rotate=270},
	shapeTriangle9/.style={shapeTriangle,fill=colorwaveG,shape border rotate=315},
}

\tikzset{%
	shapeUgaritic/.style={draw,shape=regular polygon,fill=colorwaveD,circular drop shadow,regular polygon sides=3,minimum size=\shapeSmall,inner sep=0pt,outer sep=0pt},
}

\tikzset{%
	shapeSquare/.style={draw,shape=regular polygon,fill=colorwaveC,circular drop shadow,regular polygon sides=4,minimum size=\shapeSmall,inner sep=0pt,outer sep=0pt},
	shapeSquare2/.style={shapeSquare,shape border rotate=45},
}

\tikzset{%
	shapeHexagon/.style={draw,shape=regular polygon,fill=colorwaveA,circular drop shadow,regular polygon sides=6,minimum size=\shapeSmall,inner sep=0pt,outer sep=0pt},
	shapeHexagon2/.style={shapeHexagon,shape border rotate=90},
}

\tikzset{%
	shapeOctagon/.style={draw,shape=regular polygon,fill=colorwaveB,circular drop shadow,regular polygon sides=8,minimum size=\shapeSmall,inner sep=0pt,outer sep=0pt},
	shapeOctagon2/.style={shapeHexagon,shape border rotate=45},
}
\tikzset{%
	shapeEllipse/.style={draw,shape=ellipse,minimum size=\shapeSmall,inner sep=0pt,outer sep=0pt},
	shapeEllipse2/.style={shapeEllipse,shape border rotate=90},
}

\tikzset{%
	closedFigure/.style={draw=\draw[->,rounded corners=0.2cm,shorten >=2pt]
		(1.5,0.5)-- ++(0,-1)-- ++(1,0)-- ++(0,2)-- ++(-1,0)-- ++(0,2)-- ++(1,0)--
		++(0,1)-- ++(-1,0)-- ++(0,-1)-- ++(-2,0)-- ++(0,3)-- ++(2,0)-- ++(0,-1)--
		++(1,0)-- ++(0,1)-- ++(1,0)-- ++(0,-1)-- ++(1,0)-- ++(0,-3)-- ++(-2,0)--
		++(1,0)-- ++(0,-3)-- ++(1,0)-- ++(0,-1)-- ++(-6,0)-- ++(0,3)-- ++(2,0)--
		++(0,-1)-- ++(1,0)}
}

\tikzstyle{start}=[circle, draw=none,,minimum size=\shapeMedium, fill=blue, circular drop shadow,text centered, anchor=north, text=white]
\tikzstyle{finish}=[circle, draw=none,,minimum size=\shapeMedium, fill=blue,circular drop shadow,text centered, anchor=north, text=white]
\tikzstyle{finish}=[rectangle, draw=none, ,minimum size=\shapeMedium,fill=blue,circular drop shadow,text centered, anchor=north, text=white]

\usepackage[noadjust]{cite}
\usepackage{algpseudocode}
\usepackage{listings}
\usepackage{algorithm}
\usepackage{color}
\usepackage{parskip}
\usepackage{amsfonts}
\usepackage{amsthm}
\usepackage{tikz}
\usepackage{tkz-berge}
\usepackage{caption}
\usepackage{hyperref}
\usepackage{amsrefs}
\usepackage{mathtools, amssymb}
\usepackage{graphicx}
\usepackage{subcaption}
\usepackage{tabularx,ragged2e}
\usepackage[framemethod=tikz]{mdframed}
\newcommand{\N}{\mathbb N}
\newcommand{\Q}{\mathbb Q}
\theoremstyle{definition}
\newtheorem{definition}{Definition}[section] % definitions are numbered according to sections
\newtheorem{theorem}{Theorem}[section]
\newtheorem{example}{Example}[section]
\renewcommand{\qedsymbol}{$\blacksquare$}
\newtheorem{corollary}{Corollary}[theorem]
\newtheorem{lemma}[theorem]{Lemma}

\renewcommand{\rmdefault}{ptm} 

\graphicspath{{figures/}}

\begin{document}
	
Here is a list of R packages:
	
\begin{enumerate} \itemsep -2pt
	\item  R Language \cite{key1}
	\item KEGGGraph 1 \cite{key2}
	\item KEGGGraph 2 \cite{key3}
	\item deTestSet  \cite{key4}
	\item deSolve \cite{key5}
	\item Reactive transport Modeling \cite{key6}
	\item Ecological Modeling \cite{key7}
	\item RootSolve \cite{key8}
	\item PearsonDS  \cite{key9}
	\item igraph  \cite{key10}
\end{enumerate}


\bibliographystyle{plain}
\begin{thebibliography}{00}
		
\bibitem[1]{key1}R Core Team (2015). 
\newblock R: A language and environment for statistical computing. R Foundation for Statistical Computing, Vienna, Austria.
\newblock URL https://www.R-project.org/.

\bibitem[2]{key2} Jitao David Zhang and Stefan Wiemann (2009)
\newblock  \emph{KEGGgraph: a graph approach to KEGG PATHWAY in R and Bioconductor}.
\newblock Bioinformatics, 25(11):1470--1471

\bibitem[3]{key3} Jitao David Zhang (2015). 
\newblock KEGGgraph: Application Examples
\newblock R package version 1.28.0.

\bibitem[4]{key4} Karline Soetaert, Jeff Cash and Francesca Mazzia (2016). 
\newblock deTestSet: Testset for Differential Equations. R package version 1.1.3.
\newblock http://CRAN.R-project.org/package=deTestSet

\bibitem[5]{key5}Karline Soetaert, Thomas Petzoldt, R. Woodrow Setzer (2010). 
\newblock Solving Differential Equations in R: Package deSolve. 
\newblock Journal of Statistical Software, 33(9), 1--25. URL http://www.jstatsoft.org/v33/i09/ DOI 10.18637/jss.v033.i09

\bibitem[6]{key6}Soetaert, Karline and Meysman, Filip, (2012).
\newblock Reactive transport in aquatic ecosystems: Rapid model prototyping in the open source software R
\newblock Environmental Modelling and Software, 32, 49-60.

\bibitem[7]{key7}Soetaert K. and P.M.J. Herman (2009).  
\newblock A Practical Guide to Ecological Modelling. 
\newblock Using R as a Simulation Platform.  Springer, 372 pp.

\bibitem[8]{key8}Soetaert K. (2009).  
\newblock rootSolve: Nonlinear root finding, equilibrium and steady-state analysis of ordinary differential equations.  
\newblock R-package version 1.6.

\bibitem[9]{key9}Martin Becker and Stefan Klößner (2016). 
\newblock PearsonDS: Pearson Distribution System. 
\newblock R package version 0.98. http://CRAN.R-project.org/package=PearsonDS

\bibitem[10]{key10}Csardi G, Nepusz T (2006).
\newblock The igraph software package for complex network research,
\newblock  InterJournal, Complex Systems 1695. 2006. http://igraph.org

		
\end{thebibliography}
	
\end{document}









